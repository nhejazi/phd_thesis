\usepackage{hyperref}
\hypersetup{colorlinks,citecolor=blue,urlcolor=blue}

% load biblatex after hyperref
\usepackage[natbib]{biblatex}
\bibliography{references}

%% Use the graphics package to include figures
\usepackage{graphicx}
\graphicspath{ {figs/} }
\usepackage{chngcntr}
\counterwithout{figure}{chapter}

%Interligne (pour la version finale: 1.15 parait un bon compromis)
\renewcommand{\baselinestretch}{1.15}
% caption (pris dans /usr/local/lib/tex/inputs/book.sty)
\makeatletter
\long\def\@makecaption#1#2{
   \vskip 11pt
   \setbox\@tempboxa\hbox{\parbox{12cm}{\footnotesize \hspace*{0.5cm}
    {\bf #1:}~~{\sl #2}}}
   \ifdim \wd\@tempboxa >\hsize   % IF longer than one line:
       \unhbox\@tempboxa\par      %   THEN set as ordinary paragraph.
     \else                        %   ELSE  center.
       \hbox to\hsize{\hfil\box\@tempboxa\hfil}
   \fi}
\makeatother

\let\procedure\relax
\let\endprocedure\relax
\usepackage[ruled,vlined]{algorithm2e}

% general-purpose packages
\usepackage{float}
\usepackage[english]{babel}
\usepackage{dsfont}
\usepackage{lmodern}
\usepackage[inline,shortlabels]{enumitem}
\usepackage{caption}
\usepackage{subcaption}
\usepackage{refcount}
\usepackage{color}
\usepackage{booktabs}
\usepackage{tikz}
\usepackage{listings}
\usepackage{setspace}
\usepackage{multirow}
\usepackage[T1]{fontenc}
\usepackage[utf8]{inputenc}
\usepackage{textcomp}
\usepackage{xurl}
\usepackage{csquotes}

% math packages
\usepackage{times}
\usepackage{bm}
\usepackage{bbm}
\usepackage{amsmath}
\usepackage{amsthm}
\usepackage{amsxtra}
\usepackage{commath}
\usepackage{mathtools}
\usepackage{amsfonts}
\usepackage{amssymb}

% custom math definitions and macros that conflict with Biometrika's
\newtheorem{theorem}{Theorem}
\newtheorem{lemma}{Lemma}
\newtheorem{assumption}{Assumption}

% math macros
\newtheorem{coro}{Corollary}
\DeclareMathOperator{\opt}{opt}
\DeclareMathOperator{\dr}{IF}
\newcommand{\hopt}{\hat h_{\opt}}
\newcommand{\supp}{\mathop{\mathrm{supp}}}
{\theoremstyle{definition} \newtheorem{assumptioniden}{}}
\renewcommand\theassumptioniden{{A}\arabic{assumptioniden}}
\AtEndDocument{\refstepcounter{theorem}\label{finalthm}}
\AtEndDocument{\refstepcounter{equation}\label{finaleq}}
\DeclareMathOperator{\expit}{expit}
\DeclareMathOperator{\bern}{Bern}
\DeclareMathOperator{\logit}{logit}
\DeclareMathOperator{\var}{Var}
\DeclareMathOperator{\Rem}{Rem}
\newcommand{\pt}{\mbox{$p_0$}}
\renewcommand{\P}{\mathsf{P}}
\newcommand{\m}{\mathsf{m}}
\newcommand{\p}{\mathsf{p}}
\newcommand{\q}{\mathsf{q}}
\renewcommand{\r}{\mathsf{r}}
\renewcommand{\b}{\mathsf{b}}
\renewcommand{\d}{\mathsf{d}}
\newcommand{\g}{\mathsf{g}}
\newcommand{\h}{\mathsf{h}}
\newcommand{\e}{\mathsf{e}}
\newcommand{\uu}{\mathsf{u}}
\newcommand{\vv}{\mathsf{v}}
\newcommand{\s}{\mathsf{s}}
\newcommand{\indep}{\mbox{$\perp\!\!\!\perp$}}
\newcommand{\rs}{R}
\newcommand{\ds}{D^\dag}
\newcommand{\dd}{\mathrm{d}}
\newcommand{\Pnj}{\mathsf{P}_{n,j}}
\newcommand{\Pn}{\mathsf{P}_n}
\newcommand{\Gnj}{\mathsf{G}_{n,j}}
\newcommand{\Gn}{\mathsf{G}_{n}}
\newcommand{\mut}{\mu_0}
\newcommand{\psii}{\psi_{\mbox{\scriptsize I}, \delta}}
\newcommand{\psid}{\psi_{\mbox{\scriptsize D}, \delta}}
\newcommand{\psios}{\hat\psi^{\mbox{\scriptsize os}}_\delta}
\newcommand{\psitmle}{\hat\psi^{\mbox{\scriptsize tmle}}_\delta}
\newcommand{\psidos}{\hat\psi_{\mbox{\scriptsize D},
  \delta}^{\mbox{\scriptsize os}}}
\newcommand{\psiios}{\hat\psi_{\mbox{\scriptsize I},
  \delta}^{\mbox{\scriptsize os}}}
\newcommand{\psidtmle}{\hat\psi_{\mbox{\scriptsize D},
  \delta}^{\mbox{\scriptsize tmle}}}
\newcommand{\psiitmle}{\hat\psi_{\mbox{\scriptsize I},
  \delta}^{\mbox{\scriptsize tmle}}}
\newcommand{\thetaos}{\hat\theta_{\mbox{\scriptsize os}}(\delta)}
\newcommand{\thetatmle}{\hat\theta_{\mbox{\scriptsize tmle}}(\delta)}
\newcommand{\thetaaipw}{\hat\theta_{\mbox{\scriptsize aipw}}(\delta)}
\newcommand{\hgd}{\hat g_\delta}
\newcommand{\one}{\mathds{1}}
\newcommand{\R}{\mathbb{R}}
\renewcommand{\rmdefault}{ptm}
\newcommand{\E}{\mathbb{E}}
\newcommand{\M}{\mathcal{M}}
\newcommand{\1}{\mathbbm{1}}
\newcommand{\prob}{\mathbb{P}}
\renewenvironment{proof}{{\it Proof }}{\qed \\}
\DeclareMathOperator*{\argmin}{\arg\!\min}
\def\independenT#1#2{\mathrel{\rlap{$#1#2$}\mkern2mu{#1#2}}}

\pgfdeclarelayer{background}
\pgfsetlayers{background,main}
\usetikzlibrary{arrows,positioning}
\tikzset{
%Define standard arrow tip
>=stealth',
%Define style for boxes
punkt/.style={
rectangle,
rounded corners,
draw=black, very thick,
text width=6.5em,
minimum height=2em,
text centered},
% Define arrow style
pil/.style={
->,
thick,
shorten <=2pt,
shorten >=2pt,}
}
\newcommand{\Vertex}[2]% pos, name
{\node[minimum width=0.6cm,inner sep=0.05cm] (#2) at (#1) {$\footnotesize#2$};
% \node[circle,draw,minimum width=0.6cm,inner sep=0] (#2) at (#1) {};
% \node[rounded corners=3pt,below,draw=black,fill=white,inner sep=1.5pt] at (#2.south) {\footnotesize#2};
}
\newcommand{\Vertexr}[2]% pos, name
{\node[rectangle, draw, minimum width=0.6cm,inner sep=0.05cm] (#2) at (#1) {$\footnotesize#2$};
% \node[circle,draw,minimum width=0.6cm,inner sep=0] (#2) at (#1) {};
% \node[rounded corners=3pt,below,draw=black,fill=white,inner sep=1.5pt] at (#2.south) {\footnotesize#2};
}
\newcommand{\ArrowR}[3]%
{ \begin{pgfonlayer}{background}
\draw[->,#3] (#1) to[bend right=30] (#2);
\end{pgfonlayer}
}
\newcommand{\ArrowL}[3]%
{ \begin{pgfonlayer}{background}
\draw[->,#3] (#1) to[bend left=45] (#2);
\end{pgfonlayer}
}
\newcommand{\EdgeL}[3]%
{ \begin{pgfonlayer}{background}
\draw[dashed,#3] (#1) to[bend right=-45] (#2);
\end{pgfonlayer}
}

\newcommand{\Arrow}[3]%
{ \begin{pgfonlayer}{background}
\draw[->,#3] (#1) -- +(#2);
\end{pgfonlayer}
}

% typesetting code using listings
\lstset{language=R,
     basicstyle=\ttfamily\small,
     keywordstyle=\ttfamily\small,
     stringstyle=\color{red}\ttfamily\small,
     commentstyle=\color{magenta}\ttfamily\small,
    breaklines=true
}
