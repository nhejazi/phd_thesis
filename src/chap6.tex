\chapter{Directions for Future Investigation}\label{six}

While the presented body of work has provided novel statistical methodology for
working with stochastic treatment regimes, there remains much more to do. To
motivate future efforts, there is demand for the development of theory, methods,
and software for (1) higher-order efficient estimation of novel causal effects,
including (in)direct effects; and (2) generalizable, nonparametric estimation of
flexible, path-specific causal effects for time-to-event outcomes and
outcome-dependent sampling. Such statistical developments are well-positioned to
be guided by the pressing scientific necessity of analyzing Phase III COVID-19
vaccine efficacy trials, though myriad other applications will surely emerge as
the proposed new class of methodology reaches maturity.

\section{Pressing New Challenges}

Fueled by the deluge of ``Big Data'' amassed across collaborative research
networks, observational studies arm today's scientists with the means to
investigate elaborate, albeit daunting, questions. Fortunately, formal causal
inference frameworks outline tools for expressing such questions in terms of
hypothetical interventions and complex \textit{path-specific}
effects~\citep{pearl2009causality,hernan2021causal}. Consider, for example,
efficacy trials to evaluate candidate preventive vaccines for COVID-19; the
observed data are a sample of random variables $O$, with distribution $P_0$,
where $O = (W, A, Z, M, Y)$ consists of baseline covariates $W$ (e.g., age, sex,
co-morbidities), randomized vaccine v.~placebo assignment $A$, mediating
immunologic biomarkers such as binding or neutralizing antibody levels $M$,
mediator-outcome confounders affected by treatment $Z$ (e.g., post-vaccination
risky behavior), and symptomatic SARS-CoV-2 infection $Y$. Two distinct pathways
reveal the direct and indirect effects~\citep{vanderweele2015explanation} of
vaccination: (1) $A \rightarrow Y$, and (2) $A \rightarrow M \rightarrow Y$ for
the partial effect through immunologic biomarkers. Causal inference empowers us
to investigate how disease risk \textit{would have changed} under counterfactual
treatment regimens, defined by interventions setting $A = a$ to $a^{\star} \in
\mathcal{A}$, that propagate to mediators $M$ to yield $M(a^{\star}) \in
\mathcal{M}$. These counterfactual random variables (i.e., $a^{\star}$ and
$M(a^{\star})$), with distribution $P_0^{\star}$, may be used to define highly
interpretable causal estimands $\psi_0^{\star} = \psi(P_0^{\star})$ --- for
example, the natural indirect effect $\psi_0^{\star}
= \mathbb{E}[Y(M(a^{\star}), a^{\star}) - Y(M(a),
a^{\star})]$~\citep{robins1992identifiability}, quantifying the causal impact of
vaccination on disease risk \textit{mediated only through} the immunologic
biomarkers $M$.

Although observational studies have led to a renaissance in statistical causal
inference, many significant challenges remain: causal estimands can be
``unidentifiable'' (i.e., learning $\psi(P_0^{\star})$ may be impossible from
observed data $O$), under even mildly non-standard conditions; unfounded
parametric assumptions can cause estimation bias, leading to questionable
conclusions; and, estimation of the statistical parameter $\psi(P)$ may be
complicated by biased sampling and high-dimensional or longitudinal data. Even
effect definitions can prove limiting: mediation analysis has focused on natural
(in)direct effects, whose identification assumptions preclude intermediate
confounding by $Z$~\citep{petersen2006estimation, tchetgen2014identification,
miles2017partial, vanderweele2017mediation}. While data adaptive regression may
address parametric model estimation bias, its use necessitates augmented
estimators~\citep{pfanzagl1985contributions, vdl2011targeted, vdl2018targeted,
coyle2020targeted} and mathematically intricate
derivations~\citep{carone2018higher} on a case-by-case basis for novel causal
estimands, motivating the dissemination of robust statistical software.
Real-world study designs, such as outcome-dependent two-phase sampling (the norm
in immunologic biomarker correlates analyses~\citep{haynes2012immune}), require
substantial corrections to ensure efficient effect
estimation~\citep{hejazi2020efficient} and extensions for complex (e.g.,
possibly right-censored time-to-event) outcomes.

Vaccine efficacy trials present unlimited yet pressing challenges, from a dire
need for path-specific causal estimands robustly identifiable and estimable with
right-censored mediators and outcomes to efficiency theory for estimation under
novel two-phase sampling designs.

Grounding future research efforts, we briefly review the methodological
contributions made thus far, which have leveraged advances in
non/semi-parametric causal inference to develop improved statistical estimation
techniques as well as new effect definitions extending causal inference
approaches to non-standard yet real-world settings.

\section{Efficient Estimation and Two-Phase Sampling Corrections}

Despite significant strides in statistical causal inference with the development
of multiply robust non/semi-parametric techniques (e.g., double machine
learning~\citep{chernozhukov2017double, chernozhukov2018double}, targeted
minimum loss estimation~\citep{vdl2006targeted, vdl2011targeted,
vdl2018targeted}), classical procedures such as inverse probability weighting
(IPW)~\citep{horvitz1952generalization, rosenbaum1983central, hernan2021causal}
remain widely used for their ease of implementation across a diversity of
settings~\citep{hernan2021causal}. Though straightforward, IPW estimators
require consistent estimation of a re-weighting mechanism $\g_0(A \mid W)$,
used to induce balance on $A$. Estimates $\g_n(A \mid W)$ are usually obtained
via finite-dimensional parametric models, a practice prone to biasing estimates
through model misspecification. While machine learning provides a potential
solution, most approaches are inconsistent with efficiency theory, which
requires a fast convergence rate ($1/\sqrt[4]{n}$) of the estimate to the true
mechanism (typically unattainable by machine learning). We have formulated
a novel class of IPW estimators~\citep{ertefaie2020nonparametric} by coupling
the highly adaptive lasso~\citep{benkeser2016highly, vdl2017generally,
bibaut2019fast, hejazi2020hal9001} with sieve-based selection strategies to
achieve guarantees of consistency and non/semi-parametric efficiency without
sacrificing flexible estimation of $\g_n(A \mid W)$. Our work is the first
demonstration that IPW estimators can successfully incorporate flexible modeling
without compromising asymptotic bias or variance, outlining numerous avenues for
future investigation. A simple extension of this approach appeared in
Chapter~\ref{one}, in which we developed nonparametric-efficient IPW estimators
for the causal effects of stochastic interventions, showing that this class of
procedure could work even when $\g_0$ is the conditional density of treatment,
instead of the much simpler conditional probability of treatment studied
previously.

While non/semi-parametric causal inference can provide interpretable estimands,
their estimation is often complicated by the thorny challenges posed by complex
study designs. For example, immune correlates analyses of vaccine efficacy
trials commonly use outcome-dependent two-phase
sampling~\citep{haynes2012immune}, in which immunologic biomarkers are measured
on a subset of study participants based on outcome status (e.g., infection) and
on demographic characteristics, severely complicating the statistical estimation
problem. \citet{janes2017higher} followed just such a design in providing the
first immune correlates analysis of the HIV Vaccine Trial Network 505 preventive
HIV-1 vaccine efficacy trial~\citep{hammer2013efficacy}. Seeking to quantify the
causal impact of post-vaccination antibody and T-cell immunologic biomarkers on
the risk of HIV-1 infection, in Chapter~\ref{two}
and~\citet{hejazi2020efficient}, we fused two disparate literatures on (1)
stochastic interventions~\citep{stock1989nonparametric,diaz2012population}, to
adopt a counterfactual framework applicable to continuous-valued immunologic
biomarkers, and (2) non/semi-parametric efficiency for two-phase
designs~\citep{breslow2003large,rose2011targeted2sd}, to generalize our causal
effect estimates beyond the second-phase sample. Our approach allowed for the
mean counterfactual HIV-1 infection risk in the HVTN 505 trial to be quantified
under hypothetical shifts in the observed immunologic biomarker profile. We
developed non/semi-parametric estimators of such causal effect estimands, with
guarantees of efficiency and multiple robustness despite two-phase sampling.
Using marginal structural models~\citep{neugebauer2007nonparametric}, we
formulated a dose-response analysis strategy for summarizing the effects of
shifting immunogenicity along a pre-specified grid, providing a variable
importance measure for ranking immunologic biomarkers as study endpoints in
future vaccine trials. In Chapter~\ref{three}, we showed that an extension of
this strategy could be used to study candidate mechanistic correlates of
protection in vaccine efficacy trials of COVID-19.

\section{Robust Causal Mediation for Complex Path Analysis}

While interest in path-specific effects dates back to the work
of~\citet{wright1934method}, causal mediation analysis came into its modern form
only with the definition and identifiability of the natural (in)direct
effects~\citep{robins1992identifiability, pearl2001direct}. Though the natural
(in)direct effects were significant advances, they face several limitations:
they cannot accommodate continuous-valued treatments (i.e., $A \in \{0, 1\}$);
they are defined by \textit{static} interventions, which deterministically set
$A$ to a fixed value $a$ in its support $\mathcal{A}$; their identification
requires a restriction on ``cross-world'' counterfactuals of the mediators $M$,
which is incompatible with randomization; and they are unidentifiable in the
presence of mediator-outcome confounders $Z$ affected by treatment $A$
(``intermediate confounders''). All this significantly limits the scope of
application of these canonical causal effects. To combat these shortcomings, we
developed, in~\citet{diaz2020causal}, a path-specific decomposition of the
population intervention effect~\citep{diaz2012population} of flexible,
stochastic interventions~\citep{stock1989nonparametric, diaz2013assessing,
kennedy2019nonparametric}, accommodating multivariate $M$ and applicable to
categorical or continuous $A$. Our novel (in)direct effects do not require any
cross-world counterfactual independencies, allowing for their implications to be
rigorously tested in RCTs. We outlined modern, multiply robust estimators of our
(in)direct effects (implemented in our open source \texttt{medshift} software
package~\citep{hejazi2020medshift}), alongside non/semi-parametric efficiency
theory.

Though our developments laid the foundations of a new, general framework for
causal mediation analysis, our (in)direct effects, like their natural effect
counterparts, could not accommodate intermediate confounders $Z$, motivating
work on ``interventional'' (in)direct effects~\citep{diaz2020nonparametric,
hejazi2020nonparametric}. Inteventional (in)direct
effects~\citep{vanderweele2014effect, vansteelandt2017interventional} are a new
class of causal estimands utilizing joint static and stochastic interventions
(to $A$ and $M$, respectively) to remain identifiable under intermediate
confounding. In our initial contribution to the literature on interventional
effects, we provided the first detailed developments of efficiency theory and
non/semi-parametric efficient estimators for existing interventional (in)direct
effects~\citep{diaz2020nonparametric} (and the first open source software
package, \texttt{medoutcon}, for their application~\citep{hejazi2021medoutcon}).
We then expanded upon our new mediation analysis framework by defining the
\textit{stochastic} interventional (in)direct
effects~\citep{hejazi2020nonparametric}, applicable to categorical or continuous
$A$ and identifiable under less restrictive assumptions than their classical
counterparts. It remains to generalize such effects to accommodate censored
mediators and (possibly time-to-event) outcomes.

\section{Higher-Order Efficient Estimation of Novel Causal Effects}

Several important developments must be pursued in order for the currently
budding study of higher-order efficiency to apply to complex causal effect
estimands and their non/semi-parametric estimators.
\begin{enumerate}[label=(\alph*)]
  \itemsep0.2pt
  \item Theoretical development and evaluation of novel higher-order efficient
    estimators based on targeted undersmoothing with the highly adaptive lasso
    regression function.
  \item Higher-order non/semiparametric efficient estimators of the causal
    direct and indirect effects of stochastic treatment regimes.
  \item Techniques for higher-order efficient estimation under outcome-dependent
      two-phase sampling designs, readily applicable to immune correlates
      analyses of COVID-19 and HIV-1 vaccine efficacy trials.
\end{enumerate}

Higher-order efficient estimation of the complex statistical functionals arising
in non/semi-parametric causal inference is an area largely undeveloped, though
the first steps advancing this research program are slowly being
taken~\citep{robins2008higher,carone2018higher}. Such techniques are desirable
for the fact that they expand the nuisance parameter configurations under which
multiply robust estimators may attain the non/semi-parametric efficiency bound.
In our work constructing nonparametric-efficient IPW
estimators~\citep{ertefaie2020nonparametric}, we demonstrated that a targeted
undersmoothing approach based on the highly adaptive lasso (HAL) could solve
a critical component of the estimating equation implied by the efficient
influence function (EIF). A significant step in developing this family of
approaches for higher-order efficient estimation requires deriving second-order
remainder terms $R_2(P, P_0)$ of the EIF Taylor expansion, for example, in
problems with flexible stochastic interventions on the treatment $A$ and/or of
complex causal (in)direct effect estimands. The formulation of a unique targeted
undersmoothing approach to construct HAL nuisance parameter estimators solving
the EIF estimating equation and remainder $R_2(P, P_0)$ would constitute
a generalizable strategy for higher-order estimation without reliance on
specialized knowledge beyond standard non/semi-parametric efficiency theory,
a significant step past similar recent developments~\citep{vdl2017generally,
benkeser2017doubly}. Like related approaches, our estimators would be equipped
with multiply robust confidence intervals, allowing for valid, efficient
statistical inference even under nuisance parameter misspecification. Extending
our undersmoothing approach to the augmented EIFs implied by outcome-dependent
two-phase sampling corrections would allow our higher-order estimators to be
applied in statistical analyses of immune correlates in vaccine efficacy trials,
through our ongoing collaborations with the COVID-19 Prevention Trials and HIV
Vaccine Trials Networks.

\section{Extending Causal Mediation Analysis to Complex Settings}

Causal mediation analysis has proven to be a powerful tool for dissecting the
mechanism of particular interventions in well-studied systems; however, the
complex data produced by modern studies is largely incompatible with the
comparatively simple estimation strategies prominent in mediation analysis.
Consequently, several important developments ought to be considered.
\begin{enumerate}[label=(\alph*)]
  \itemsep0.2pt
  \item Novel decompositions of direct and indirect causal effects as well as
      their efficient estimation in settings with possibly right-censored
      time-to-event outcomes and outcome-dependent two-phase sampling.
  \item Robust, novel causal inference transport methodology for bridging these
    direct and indirect effects to heterogeneous populations (e.g., between
    vaccine efficacy trials).
  \item Open source software for the adoption of these newly formulated direct
    and indirect effects in immune correlates analyses of COVID-19 and HIV-1
     vaccine efficacy trials.
\end{enumerate}

Biomedical and health studies, vaccine efficacy trials especially, are often
complicated by time-to-event outcomes and outcome-dependent two-phase
sampling~\citep{follmann2006augmented} (e.g.,
case-cohort~\citep{prentice1986case, barlow1999analysis, mcelrath2008hiv}).
A common goal is to ascertain the effect of a treatment, through mediating
variables, on the occurrence of a possibly right-censored time-to-event outcome.
For example, in immune correlates analyses of vaccine
trials~\citep{haynes2012immune}, the observed data are random variables $X =
(W, A, Z, M, \Delta, \widetilde{T})$, where $\widetilde{T} = \min(T, C)$ for
failure time $T$ and censoring time $C$ with $\Delta = \mathbb{I}(T \leq C)$
indicating observed failure (other variables remain as previously defined). To
assess the indirect effect of vaccination $A$ through immunologic biomarkers
$M$, two innovations are necessary: (1) new path-specific effects identifiable
under intermediate confounding and capable of handling time-to-event outcomes,
and (2) efficient estimation strategies for outcome-dependent two-phase sampling
based on $\widetilde{T}$. For (1), we have proposed flexible (in)direct effect
estimands~\citep{diaz2020causal, hejazi2020nonparametric}, robust to
intermediate confounding; however, our (in)direct effects accommodate neither
time-to-event outcomes nor censoring. The study of (in)direct effect estimands
under such conditions is in its infancy: current
strategies~\citep[e.g.,][]{tchetgen2011causal, zheng2017longitudinal} break down
under intermediate confounding or are limited to binary treatments $A \in \{0,
1\}$. For (2), current methods construct second-phase samples based on $\Delta$
rather than $\widetilde{T}$, a limitation ignoring the time-to-event nature of
the outcome process and constraining sampling efficiency. Generalizing
(in)direct effect estimands necessitates methods for bridging estimates to new
populations (i.e., external validity); such \textit{causal transport}
techniques~\citep{pearl2011transportability, pearl2014external,
bareinboim2016causal} are under study but variants for
path-specific effects~\citep[e.g.,][]{rudolph2020efficiently}
are in a nascent stage. We will complement our theoretical and methodological
pursuits by developing and disseminating open source statistical software and
with the application of these approaches in immune correlates analyses of Phase
III vaccine trials through our collaborative role in the COVID-19 Prevention
Trials Network.

\section{A Broader Significance: Advancing Science}

Since path-specific causal effects are ideal for developing scientific answers
to questions of mechanism, my proposed research program will have important
bearings on problems across many fields, from epidemiology and medicine to
economics and psychology. In the health and medical sciences, where the
measurement of important but costly variables (e.g., immunologic biomarkers)
complicates effect estimation, sophisticated methods for handling
outcome-dependent two-phase sampling, a key tool in scientific trial design, are
sorely needed. I am particularly motivated by avenues for formulating direct and
indirect effects applicable to complex data from vaccine efficacy trials and
will leverage my collaborations with the COVID-19 Vaccine Prevention Trials and
HIV Vaccine Trials Networks to inform scientifically impactful open problems in
causal inference. In turn, the methods we develop will help to maximize what we
can learn about critical and timely scientific questions, such as how best to
tailor future vaccines to mitigate infection risk.
