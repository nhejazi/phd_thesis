\chapter{Future Work}

The proposed procedure for applying empirical Bayes moderated statistics to
asymptotically linear parameters currently suffers from several limitations,
both in terms of pragmatic theoretical development and software implementation.
As previously discussed, empirical Bayes moderation of estimates of parameters
with asymptotically linear representations may be used to obtain robust standard
deviation estimates when the estimated parameter is formulated in terms of a
binary exposure variable. It is for this reason that we largely rely on the
moderated t-statistic of Smyth~\cite{smyth2004linear} in our presentation of
applied data analysis. The limitation imposed by assessing the difference
between two levels of the exposure variable has several solutions: (1) the
applied researcher could decide on a binarization scheme that fits their
scientific question of interest, and (2) the range of target parameters
currently supported by our software implementation~\cite{hejazi2017biotmle}
could be increased beyond just the average treatment effect.

Going further, this application of moderated statistics to the development of
robust hypothesis testing procedures for asymptotically linear parameters may be
further extended to a general notion of hypothesis testing for parameters that
do not have straightforward asymptotically linear representations. Specifically,
one might be interested in assessing more complex parameters (e.g., the causal
dose-response curve) that allow a greater degree of flexibility in answering
scientific questions of interest --- that is, it is easy to conceive of
investigations in which a dose-response relationship is of interest, rather than
a mere difference in the levels of a binary exposure. Supporting the application
of moderated statistics to such parameters would greatly increase the
flexibility of the proposed procedure.

While the proposed method has been well-established theoretically, there are a
number of computational improvements that can still be made, including both
improvements of the software to meet the standards of the Bioconductor project
and numerical simulations to confirm that the application of moderated
statistics to asymptotically linear estimates of target parameters of interest
produces consistent results. Several ideas for simulation studies have been
proposed, and work is underway to implement these ideas, to ensure that the
proposed method and corresponding software implementation do hold to the
theoretical properties presented in preceding sections.
