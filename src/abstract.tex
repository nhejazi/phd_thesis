% (This file is included by thesis.tex; you do not latex it by itself.)

\begin{abstract}

Nearly a century ago, the foundations of modern statistics laid the groundwork
for a science of causality. Today, causal inference is central to the study of
the most impactful questions at the intersection of science and policy: By what
mechanisms do novel therapeutics mitigate relapse in addiction disorders? How do
immunobiological markers mediate action mechanisms of vaccines? While
randomization provides ``gold standard'' tools for quantifying causal effects,
such trials are costly and limit the scope of scientific inquiry. Thus,
techniques for statistical causal inference with complex, observational data are
critical to today's, and tomorrow's, scientific endeavors.

Observational studies obviate many of the shortcomings of randomized trials but
bring their own challenges and promises. Without randomization, causal inference
is plagued by confounding: vaccinees may be more likely to engage in risky
behaviors and patients assigned a candidate therapeutic are not uniformly
``treated'' due to clinician heterogeneity. Adjusting for potential confounders
is a daunting challenge in an era where studies routinely measure numerous
high-dimensional, longitudinal characteristics. Further, observational studies
empower scientists to assess mechanistic, path-specific causal effects that
\textit{cannot be learned} with randomized data. Tools from non/semi-parametric
statistical theory and machine learning are needed to avoid imposing unrealistic
statistical assumptions, and novel causal effect estimands are required to
better address mechanistic questions.

Causal inference methodology is critical to answering real-world scientific
questions, but traditional approaches make too many simplifying assumptions. By
ignoring biased sampling designs, continuous-valued (or ``quantitative'')
treatments, and confounding of path-specific effects, standard statistical
methods fall far short of empowering mechanistic discovery. Such techniques
often require \textit{a priori} modeling assumptions unsupported by domain
knowledge, limiting their utility for real-world data analyses.

This thesis extends theory and methods for non/semi-parametric causal inference
in settings with quantitative treatments, with particular attention paid to
issues emerging from biased sampling designs and path-specific causal effects.
Across this range of problems, \textit{stochastic treatment regimes} are
leveraged as a unifying framework for formulating and evaluating the causal
effects of quantitative treatments. Chapter~\ref{one} considers estimation of
the generalized propensity score, a nuisance parameter critical to constructing
estimators of the causal effects of stochastic interventions. Since this
nuisance quantity is the conditional density of the treatment, given baseline
covariates, its estimation is significantly more challenging than the classical
propensity score for treatments with few levels. Towards this end, we formulate
algorithms for flexibly estimating this quantity using the highly adaptive lasso
nonparametric regression estimator; moreover, we complement these contributions
by developing novel inverse probability weighted estimators, capable of
attaining the nonparametric efficiency bound. Chapter~\ref{two} focuses on the
application of the causal effects of stochastic interventions in real-world
studies, which routinely rely upon outcome-dependent two-phase sampling designs
(e.g., case-cohort) to circumvent constraints imposed by measuring costly
biomarker variables and rare outcomes. The contribution of this work includes
a methodological advance that marries the separate literatures on causal
inference with stochastic interventions and biased sampling corrections, which
ultimately allows for complex causal parameters to be efficiently estimated
under such designs, helping to maximize what \textit{can be learned} from
critical and timely scientific inquiries, such as how best to tailor future
vaccines to mitigate infection risk. The work is motivated by the secondary aim
of an HIV vaccine efficacy trial, which probed how the vaccination-induced
immunogenicity of candidate immune correlates of protection may best be
modulated by future vaccines, and the proposed methodology is demonstrated
through a re-analysis of the data from this trial. Chapter~\ref{three} examines
path-specific causal effects (i.e., causal mediation analysis), formulated based
upon stochastic interventions, introducing a new class of direct and indirect
effect parameters that remain identifiable under intermediate confounding. The
goal of this body of work is to develop nonparametric techniques for flexible,
efficient estimation of robust path-specific effects that may be used to
quantify mechanistic knowledge and extract actional insights from the ``Big
Data'' produced by modern, large-scale studies and across a great diversity of
scientific efforts. Chapter~\ref{four} discusses open source software packages
for causal inference, which implement the statistical methodology discussed
prior. Chapter~\ref{five} concludes with a discussion of future avenues of
investigation that may motivate postdoctoral research.

\end{abstract}
