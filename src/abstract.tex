% (This file is included by thesis.tex; you do not latex it by itself.)

\begin{abstract}

Nearly a century ago, the foundations of modern statistics laid the groundwork
for a science of causality.
Today, causal inference is central to the study of the most impactful questions
at the intersection of science and policy: By what mechanisms do novel
therapeutics mitigate relapse in addiction disorders? How do immunobiological
markers mediate action mechanisms of vaccines? While randomization provides
``gold standard'' tools for quantifying causal effects, such trials are costly
and limit the scope of scientific inquiry. Thus, techniques for statistical
causal inference with complex, observational data are critical to today's, and
tomorrow's, scientific endeavors.

Observational studies obviate many of the shortcomings of randomized trials but
bring their own challenges and promises.
Without randomization, causal inference is plagued by confounding: vaccinees may
be more likely to engage in risky behaviors and patients assigned a candidate
therapeutic are not uniformly ``treated'' due to physician heterogeneity.
Adjusting for potential confounders is a daunting challenge in an era where
studies routinely measure numerous high-dimensional, longitudinal
characteristics. Further, observational studies empower scientists to assess
mechanistic, path-specific causal effects that \textit{cannot be learned} with
randomized data. Tools from non/semi-parametric statistical theory and machine
learning are needed to avoid imposing unrealistic statistical assumptions, and
novel causal effect estimands are required to better address mechanistic
questions.

This thesis extends theory and methods for
non/semi-parametric causal inference in settings with quantitative treatments,
path-specific causal effects, complex sampling designs --- all within the
framework of \textit{stochastic treatment regimes}.

Causal inference methodology is critical to answering real-world scientific
questions, but traditional approaches make too many simplifying assumptions. By
ignoring complex sampling, continuous-valued treatments, and confounding of
path-specific effects, standard methods fall far short of empowering mechanistic
discovery. Such techniques often require \textit{a priori} modeling assumptions
unsupported by domain knowledge, limiting their utility further. The fellow will
develop new theory and methods urgently required for estimation and inference in
these settings.

Path-specific causal effects arise across a diversity of scientific efforts to
quantify mechanistic knowledge. In order to utilize ``Big Data'' to develop
actionable insights, modern, large-scale studies require novel
non/semi-parametric techniques for flexible, efficient estimation of robust
path-specific effects that may be bridged between heterogeneous populations.
This is not enough, however. Real-world studies routinely rely upon
outcome-dependent two-phase sampling designs (e.g., case-cohort) to circumvent
constraints imposed by measuring costly biomarker variables and rare outcomes.
Thus, methodological advances for efficiently estimating complex parameters
under such designs are of increasingly great relevance, and their development
will help to maximize what \textit{can be learned} from critical and timely
scientific inquiries, such as how best to tailor future vaccines to mitigate
infection risk.

\end{abstract}
