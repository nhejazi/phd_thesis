% (This file is included by thesis.tex; you do not latex it by itself.)

\begin{abstract}

Nearly a century ago, the foundations of modern statistics laid the groundwork
for a science of causality. Today, causal inference is central to the study of
the most impactful questions at the intersection of science and policy: By what
mechanisms do novel therapeutics mitigate relapse in addiction disorders? How do
immunobiological markers mediate action mechanisms of vaccines? While
randomization provides ``gold standard'' tools for quantifying causal effects,
such trials are costly and limit the scope of scientific inquiry. Thus,
techniques for statistical causal inference with complex, observational data are
critical to today's, and tomorrow's, scientific endeavors.

Observational studies obviate many of the shortcomings of randomized trials but
bring their own challenges and promises. Without randomization, causal inference
is plagued by confounding: vaccinees may be more likely to engage in risky
behaviors and patients assigned a candidate therapeutic are not uniformly
``treated'' due to clinician heterogeneity. Adjusting for potential confounders
is a daunting challenge in an era where studies routinely measure numerous
high-dimensional, longitudinal characteristics. Further, observational studies
empower scientists to assess mechanistic, path-specific causal effects that
\textit{cannot be learned} with randomized data. Tools from non/semi-parametric
statistical theory and machine learning are needed to avoid imposing unrealistic
statistical assumptions, and novel causal effect estimands are required to
better address mechanistic questions.

Causal inference methodology is critical to answering real-world scientific
questions, but traditional approaches make too many simplifying assumptions. By
ignoring biased sampling designs, continuous-valued (or ``quantitative'')
treatments, and confounding of path-specific effects, standard statistical
methods fall far short of empowering mechanistic discovery. Such techniques
often require \textit{a priori} modeling assumptions unsupported by domain
knowledge, limiting their utility for real-world data analyses.

This thesis extends theory and methods for non/semi-parametric causal inference
in settings with quantitative treatments, with particular attention paid to
issues emerging from biased sampling designs and path-specific causal effects.
\textit{Stochastic treatment regimes} provide a unifying framework for
formalizing such causal inference problems. Chapter~\ref{one} considers
estimation of the generalized propensity score, a quantity critical to
estimating the causal effects of stochastic interventions. To tractably estimate
this challenging quantity, we formulate algorithms for its flexibly estimation
using the highly adaptive lasso, a nonparametric regression estimator. We then
develop novel inverse probability weighted estimators of the causal effects of
stochastic interventions, and show them capable of attaining the semiparametric
efficiency bound. Chapter~\ref{two} focuses on the application of the causal
effects of stochastic interventions in real-world studies that rely upon
outcome-dependent two-phase sampling (e.g., case-cohort designs). The work
includes a methodological advance that unites techniques for estimating the
causal effects of stochastic interventions with corrections for biased sampling,
allowing for these complex causal parameters to be efficiently estimated under
such designs. Motivated by the aims of an HIV vaccine efficacy trial, this
contribution allows researchers to probe how the vaccination-induced
immunogenicity of candidate immune correlates of protection may best be
modulated by future vaccines, and the proposed methodology is demonstrated
through a re-analysis of the data from this trial. As the COVID-19 pandemic took
the world by storm during the course of this work, Chapter~\ref{three}
generalizes the methodology proposed in the preceding chapter to help maximize
what can be learned from the critical and timely scientific inquiries posed by
COVID-19 vaccine trials. Chapter~\ref{four} examines path-specific causal
effects (i.e., causal mediation analysis), formulated based upon stochastic
interventions, introducing a new class of direct and indirect effect parameters
robust to intermediate confounding. Developing semiparametric-efficient
techniques for the flexible estimation of these path-specific effects,
facilitates their use in quantifying mechanistic knowledge and extracting
actional insights from the ``Big Data'' produced by modern, large-scale studies.
Chapter~\ref{five} discusses open source software packages for causal inference,
which implement the statistical methodology discussed prior. Chapter~\ref{six}
concludes with a discussion of avenues that may motivate future research.

\end{abstract}
